%!TEX program = xelatex
\documentclass[a4paper,10pt,twocolumn,oneside]{article}
\setlength{\columnsep}{10pt}                                                                    %兩欄模式的間距
\setlength{\columnseprule}{0pt}                                                                %兩欄模式間格線粗細

\usepackage{amsthm}								%定義,例題
\usepackage{amssymb}
%\usepackage[margin=2cm]{geometry}
\usepackage{fontspec}								%設定字體
\usepackage{color}
\usepackage[x11names]{xcolor}
\usepackage{listings}								%顯示code用的
%\usepackage[Glenn]{fncychap}						%排版,頁面模板
\usepackage{fancyhdr}								%設定頁首頁尾
\usepackage{graphicx}								%Graphic
\usepackage{enumerate}
\usepackage{changepage}
\usepackage[compact]{titlesec}    %compact mode for reducing margin
\usepackage{amsmath}
\usepackage[CheckSingle, CJKmath]{xeCJK}
\usepackage{hyperref}
\usepackage{textcomp}                               % for ' and `
\hypersetup{
  linktoc=all,
  hidelinks
}
% \usepackage{CJKulem}

%\usepackage[T1]{fontenc}
\usepackage{courier}
\topmargin=-1pt
\headsep=5pt
\textheight=780pt
\footskip=0pt
\voffset=-40pt
\textwidth=545pt
\marginparsep=0pt
\marginparwidth=0pt
\marginparpush=0pt
\oddsidemargin=0pt
\evensidemargin=0pt
\hoffset=-42pt

%\renewcommand\listfigurename{圖目錄}
%\renewcommand\listtablename{表目錄} 

%%%%%%%%%%%%%%%%%%%%%%%%%%%%%

\setmainfont{Consolas}
\setmonofont{Consolas}
\setCJKmainfont{Noto Sans CJK TC}
\XeTeXlinebreaklocale "zh"						%中文自動換行
\XeTeXlinebreakskip = 0pt plus 0pt				%設定段落之間的距離
\setcounter{secnumdepth}{3}						%目錄顯示第三層

%%%%%%%%%%%%%%%%%%%%%%%%%%%%%
\makeatletter
\lst@CCPutMacro\lst@ProcessOther {"2D}{\lst@ttfamily{-{}}{-{}}}
\@empty\z@\@empty
\makeatother
\lstset{											% Code顯示
language=C++,										% the language of the code
basicstyle=\footnotesize\ttfamily, 						% the size of the fonts that are used for the code
%numbers=left,										% where to put the line-numbers
numberstyle=\footnotesize,						% the size of the fonts that are used for the line-numbers
stepnumber=1,										% the step between two line-numbers. If it's 1, each line  will be numbered
numbersep=5pt,										% how far the line-numbers are from the code
backgroundcolor=\color{white},					% choose the background color. You must add \usepackage{color}
showspaces=false,									% show spaces adding particular underscores
showstringspaces=false,							% underline spaces within strings
showtabs=false,									% show tabs within strings adding particular underscores
frame=false,											% adds a frame around the code
tabsize=2,											% sets default tabsize to 2 spaces
captionpos=b,										% sets the caption-position to bottom
breaklines=true,									% sets automatic line breaking
breakatwhitespace=false,							% sets if automatic breaks should only happen at whitespace
escapeinside={\%*}{*)},							% if you want to add a comment within your code
morekeywords={*},									% if you want to add more keywords to the set
keywordstyle=\bfseries\color{Blue1},
commentstyle=\itshape\color{Red4},
stringstyle=\itshape\color{Green4},
literate={\ \ }{{\ }}1
}

%%%%%%%%%%%%%%%%%%%%%%%%%%%%%

\def\footnotesize{\fontsize{8}{9.5}\selectfont}
\titlespacing*{\section} {0pt}{2ex plus 1ex minus .2ex}{0.8ex plus .2ex}  % minimize margin
\titlespacing*{\subsection} {0pt}{1.75ex plus 1ex minus .2ex}{0ex plus .2ex} % minimize margin

\begin{document}
\pagestyle{fancy}
\fancyfoot{}
\fancyhead[L]{National Taiwan University - GenshinStartKleePlayer}
\fancyhead[R]{\thepage}
\renewcommand{\headrulewidth}{0.4pt}
\renewcommand{\contentsname}{Contents} 

\scriptsize
\vspace{-2em}
\tableofcontents
\vspace{-1em}

%%%%%%%%%%%%%%%%%%%%%%%%%%%%%


\newcommand{\includecode}[2]{
  \subsection{#1}
  \vspace{-0.8em}
  \lstinputlisting{#2}
  \vspace{-1.2em}
}

\newcommand{\includetex}[2]{
  \subsection{#1}
  \input{#2}
  \vspace{-1.2em}
}

\newcommand{\sectiontitle}[1]{
  \section{#1}
  \vspace{-0.5em}
}

%%%%%%%%%%%%%%%%%%%%%%%%%%%%%

\sectiontitle{Basic}

% \includecode{Default}{../codes/Basic/Default.cpp}
% \includecode{vimrc}{../codes/Basic/vimrc}
\includecode{Pragma}{../codes/Basic/Pragma.cpp}


\sectiontitle{Data Structure}

\includecode{Black Magic}{../codes/Data_Structure/Pbds.cpp}
\includecode{Lichao Tree}{../codes/Data_Structure/Lichao_Tree.cpp}
\includecode{Linear Basis}{../codes/Data_Structure/Linear_Basis.cpp}


\sectiontitle{Graph}

\includecode{Bridge CC}{../codes/Graph/Bridge_CC.cpp}
\includecode{Dinic}{../codes/Graph/Dinic.cpp}
\includecode{Min Cost Max Flow}{../codes/Graph/Min_Cost_Max_Flow.cpp}
\includecode{Stoer Wagner Algorithm}{../codes/Graph/Stoer_Wagner.cpp}
\includecode{Hopcroft Karp Algorithm}{../codes/Graph/Hopcroft_Karp.cpp}
\includecode{General Matching}{../codes/Graph/General_Matching.cpp}


\sectiontitle{Geometry}

\includecode{Basic}{../codes/Geometry/Basic.cpp}
\includecode{2D Convex Hull}{../codes/Geometry/2D_ConvexHull.cpp}


\sectiontitle{String}

\includecode{KMP}{../codes/String/Kmp.cpp}
\includecode{Suffix Array}{../codes/String/Suffix_Array.cpp}
\includecode{Booth Algorithm}{../codes/String/Booth.cpp}
\includecode{Manacher Algorithm}{../codes/String/Manacher.cpp}


\sectiontitle{Math}

\includetex{Numbers}{../codes/Math/Numbers.tex}
\includecode{Extgcd}{../codes/Math/Extgcd.cpp}
\includecode{Linear Sieve}{../codes/Math/Linear_Sieve.cpp}
\includecode{Fast Walsh Transform}{../codes/Math/Fast_Walsh_Transform.cpp}
\includecode{Floor Sum}{../codes/Math/Floor_Sum.cpp}
\includecode{Linear Programming}{../codes/Math/Linear_Programming.cpp}
\includecode{Miller Rabin}{../codes/Math/Miller_Rabin.cpp}
\includecode{Pollard's Rho}{../codes/Math/Pollard_Rho.cpp}
\includecode{Gauss Elimination}{../codes/Math/Gauss_Elimination.cpp}
\includecode{Fast Fourier Transform}{../codes/Math/Fast_Fourier_Transform.cpp}
\includecode{3 Primes NTT}{../codes/Math/Three_Primes_NTT.cpp}
\includecode{Number Theory Transform}{../codes/Math/Number_Theory_Transform.cpp}


\end{document}
